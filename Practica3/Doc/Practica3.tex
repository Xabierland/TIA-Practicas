\documentclass{report}
\usepackage[spanish]{babel}
\usepackage[utf8]{inputenc}
\usepackage{graphicx, longtable, float, titlesec, hyperref, enumitem, dingbat, soul, multicol, listings}
\usepackage[dvipsnames]{xcolor}
\usepackage[margin=2cm]{geometry}

% Cambia el color de los links
\hypersetup{
    hidelinks = true
}

% Python Code
\lstdefinestyle{Python}{
  commentstyle=\color{brown},
  keywordstyle=\color{violet},
  numberstyle=\tiny\color{gray},
  stringstyle=\color{purple},
  basicstyle=\ttfamily\footnotesize,
  breakatwhitespace=false,         
  breaklines=true,                 
  captionpos=b,                    
  keepspaces=true,                 
  numbers=left,                    
  numbersep=5pt,                  
  showspaces=false,                
  showstringspaces=false,
  showtabs=false,                  
  tabsize=2,
  literate={ñ}{{\~n}}1 {á}{{\'a}}1 {é}{{\'e}}1 {í}{{\'i}}1 {ó}{{\'o}}1 {ú}{{\'u}}1
}
\lstset{style=Python}

% Elimina la palabra "Capítulo" de los títulos de los capítulos
\titleformat{\chapter}[display]
  {\normalfont\bfseries}{}{0pt}{\Huge\thechapter.\space}

\titleformat{name=\chapter,numberless}[display]
  {\normalfont\bfseries}{}{0pt}{\Huge}

\titlespacing*{\chapter}{0pt}{-50pt}{20pt}

% Personalización del índice de listados
\renewcommand{\lstlistingname}{Código}  % Cambiar el nombre de "Listing" a "Código"
\renewcommand{\lstlistlistingname}{Índice de Códigos}

% Añade numeración a los subsubsection*s y los añade al índice
\setcounter{secnumdepth}{4}
\setcounter{tocdepth}{4}

\begin{document}
    \begin{titlepage}
        \centering
        \includegraphics[width=0.6\textwidth]{./.img/logo.jpg}\\
        \vspace{1cm}
        \LARGE Técnicas de Inteligencia Artificial\\
        \vspace{0.5cm}
        \Large Ingeniería Informática de Gestión y Sistemas de Información\\
        \vspace{3cm}
        \Huge Practica 3\\
        \huge Clasificacion\\
        \vspace{2.5cm}
        \Large Autor(es):\\
        \vspace{0.2cm}
        \large Xabier Gabiña\\
        \large Diego Montoya\\
        \vfill
        \today
    \end{titlepage}
    \tableofcontents
    \listoffigures
    \listoftables
    \lstlistoflistings
    \chapter{Introducción}
    \chapter{Ejercicios}
      \section{Perceptron}
      \section{Clonando el Comportamiento del Pacman}
      \section{Clonando el Comportamientodel Pacman con rasgos diseñados por nosotros}
    \chapter{Resultados}
      \section{Casos de prueba}
        \subsection{Perceptron}
        \subsection{Clonando el Comportamiento del Pacman}
        \subsection{Clonando el Comportamiento del Pacman con rasgos diseñados por nosotros}
      \section{Autograder}
\end{document}